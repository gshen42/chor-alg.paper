\section{A Minimal Algebraic Effects Framework in Agda}

In this section, we define a minimal algebraic effects framework in Agda, which we will use to implement CP in the next section.
%
No prior knowledge of algebraic effects is assumed, and we introduce every concept as we go.
%
Algebraic effects provide an abstraction for defining and interpreting effectful computations with a solid foundation in category theory.
%
Each concept is first introduced from a mathematical perspective and then related to programming.
%
We also use stateful computations as an example to demonstrate the use of each concept.
%
Our presentation is influenced by \citet{bauer-2019} and \citet{kidney-2023} but is minimized for our purpose.

\begin{code}[hide]
open import Data.Product using (Σ; Σ-syntax; _,_)
open import Function using (_∘_)

infix 21 ⟦_⟧
\end{code}

\subsection{Signatures and Algebras}

An effect signature \textit{Sig}, or simply signature, specifies the interface of an effectful computation, which includes a type \textit{Op} of operations and a \textit{Arity} function giving the arity of each operation:
%
\begin{center}\begin{code}
record Sig : Set₁ where
  constructor _◁_
  field
    Op : Set
    Arity : Op → Set
\end{code}\end{center}
%
\begin{code}[hide]
open Sig
\end{code}

A signature $\mathbb{F}$ induces a functor $[\![ \mathbb{F} ]\!]$ that interprets all the operations as functions on a carrier set $X$:
%
\begin{center}\begin{code}
⟦_⟧ : Sig → Set → Set
⟦ Op ◁ Ar ⟧ X = Σ[ o ∈ Op ] (Ar o → X)

_-Alg[_] : Sig → Set → Set
𝔽 -Alg[ X ] = ⟦ 𝔽 ⟧ X → X
\end{code}\end{center}

\subsection{Free Algebras}

\begin{center}\begin{code}
data Term (𝔽 : Sig) (A : Set) : Set where
  var : A → Term 𝔽 A
  op : ⟦ 𝔽 ⟧ (Term 𝔽 A) → Term 𝔽 A

perform : ∀ {𝔽} (o : Op 𝔽) → Term 𝔽 (Arity 𝔽 o)
perform o = op (o , var)
\end{code}\end{center}

\subsection{Effect Handlers}

\begin{center}\begin{code}
interp : ∀ {𝔽} {X A} → 𝔽 -Alg[ X ] → (A → X) → Term 𝔽 A → X
interp alg f (var x)      = f x
interp alg f (op (o , k)) = alg (o , interp alg f ∘ k)
\end{code}\end{center}
