\section{Next Steps}
\label{sec:next}

\begin{code}[hide]
open import 02-algeff

module 05-next (𝕃 : Sig) where

import 03-choreo
import 04-proofs

open module 03-choreo′ = 03-choreo 𝕃
open module 04-proofs′ = 04-proofs 𝕃

\end{code}

Our next step will be to leverage the algebraic-effects-based formulation of CP presented in the last two sections to prove the correctness of endpoint projection, as well as follow-on properties such as deadlock freedom.

At a high level, given a choreographic semantics \AgdaDatatype{\_⇒ᶜ\_} and a network semantics \AgdaDatatype{\_⇒ⁿ\_}, \emph{soundness} of endpoint projection would say that the projected network \emph{preserves} the semantics of the original choreography.
%
That is, for choreographies $c$ and $c'$,
%
\begin{code}[hide]
_ = ∀ {F} (c c′ : Choreo F) →
\end{code}
%
\begin{code}[inline]
  epp c ⇒ⁿ epp c′ → c ⇒ᶜ c′
\end{code}.
%
\emph{Completeness} of EPP, on the other hand, would say that the network \emph{reflects} the semantics of the original choreography, that is, given choreographies $c$ and $c'$,
%
\begin{code}[hide]
_ = ∀ {F} (c c′ : Choreo F) →
\end{code}
%
\begin{code}[inline]
  c ⇒ᶜ c′ → epp c ⇒ⁿ epp c′
\end{code}.
%
If these correctness conditions hold, the transition systems \AgdaDatatype{\_⇒ᶜ\_} and \AgdaDatatype{\_⇒ⁿ\_} are in bisimulation,
%
and if choreographies enjoy a progress property, networks also have it, implying that they are deadlock-free.
%
However, the above definitions of soundness and completeness may be too strong --- for instance, they prohibit EPP from doing rewritings that change the behaviors of the network but compute the same result.
%
Thus, we are working on more relaxed correctness conditions that permit more interesting behaviors in networks while still maintaining deadlock freedom.

In the longer term, we want to bring our algebraic-effects-based formulation of CP to languages with efficient native support for algebraic effects, such as OCaml and Koka.
