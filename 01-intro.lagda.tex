\section{Introduction}

Choreographic programming~(CP)~\citep{montesi-2013, montesi-2023} is a paradigm for programming distributed applications that run on multiple nodes.
%
In CP, instead of programming each node separately, the programmer writes one, unified program, called a \emph{choreography}, that is then compiled to individual programs for each node via a compilation step called \emph{endpoint projection} (EPP).
%
For example, one can implement a distributed pipeline that utilizes the processing power of nodes A, B, and C as a choreography:
%
\begin{equation*}
  \begin{split}
    x \leftarrow \; & \locally{A}{\mathsf{getInput}} \\
    y \leftarrow \; & \comm{A}{B}{\mathsf{process}_A(x)} \\
    z \leftarrow \; & \comm{B}{C}{\mathsf{process}_B(y)} \\
    w \leftarrow \; & \comm{C}{A}{\mathsf{process}_C(z)} \\
                    & \locally{A}{\mathsf{showResults}(w)} \\
  \end{split}
\end{equation*}
%
Here, we use $\leftarrow$ for variable bindings, $\locally{\_}{\_}$ for node-local computations, $\comm{\_}{\_}{\_}$ for communication between nodes.
%
In this choreography, A first gets some input locally, then passes it around to B and C, with each node calling its process function on the data.
%
The result is passed back to A and printed out.
%
To get the deployable implementation for each node, we can apply endpoint projection on the choreography, which generates the following individual programs for A, B, and C:
%
\[
\begin{minipage}{.3\textwidth}
  \begin{equation*}
    \begin{split}
      x \leftarrow \; & \mathsf{getInput} \\
                      & \mathsf{send}(B, \mathsf{process}_A(x)) \\
      w \leftarrow \; & \mathsf{recv}(C) \\
                      & \mathsf{showResult}(w)
    \end{split}
  \end{equation*}
\end{minipage}
\hfill\vline\hfill
\begin{minipage}{.3\textwidth}
  \begin{equation*}
    \begin{split}
      y \leftarrow \; & \mathsf{recv}(A) \\
                      & \mathsf{send}(C, \mathsf{process}_B(y)) \\
    \end{split}
  \end{equation*}
\end{minipage}
\hfill\vline\hfill
\begin{minipage}{.3\textwidth}
  \begin{equation*}
    \begin{split}
      z \leftarrow \; & \mathsf{recv}(B) \\
                      & \mathsf{send}(A, \mathsf{process}_C(z)) \\
    \end{split}
  \end{equation*}
\end{minipage}
\]

A correct CP language guarantees soundness and completeness of EPP, which further implies that the resulting programs are deadlock-free when running together.
%
Previous work has been done on formalizing a core CP calculus and proving its correctness~\citep{cruzfilipe-2020, cruzfilipe-2022, hirsch-2022}.
%
With these theoretical foundations, people have been building practical CP languages in the form of a custom compiler for Java~\citep{giallorenzo-2024} and libraries for Haskell~\citep{shen-2023} and Rust~\citep{kashiwa-2023}, and use them to build real-world distributed applications~\citep{lugovic-2023}.

Sadly, we usually don't have proofs of correctness and practical implementations at the same time.
%
Proofs are for high-level calculi that are distant from how people write code, and implementations don't have any formal guarantee of the programs they generate.
%
Recent efforts have been made to use a formal reasoning system to define and prove the correctness of CP and then extract a verified implementation~\citep{pohjola-2022, cruzfillipe-2023}.
%
However, these verified implementations fall short of being widely applicable because they often define a standalone language for CP, making it hard to integrate with existing language infrastructures.
%
In particular, these verified implementations define a deep embedding of CP in the host language with variables being in a separate semantic category, making it hard to apply normal host language functions on them without complicated marshalling and unmarshalling (\gs{Is this clear enough? I could go deeper into how their approach sucks, but not sure if it's worth the space.}).
%
On the other hand, a shallow embedding approach would not suffice either because proofs need access to the abstract syntax trees to reason about them.

To that end, we propose using a \emph{mixed embedding} approach to build a verified CP language in proof assistant Agda.
%
Our mixed embedding utilizes algebraic effects to represent CP with CP-specific effects and control flows deeply embedded --- enalbing reasoning about them --- pure computations and variable bindings shallowly embedded --- enabling seamless integration with the host language.
%
The resulting implementation can be easily either extracted from Agda or ported to a different language that supports algebraic effects without many changes.
%
Given that more and more languages are adapting algebraic effects as an abstraction for user-defined effects and some even support efficient complilation of them with performance on par with native code, we believe this approach can bring CP to a broader audience with greater confidence in its correctness.

In this paper, we outline our plan for building a verified CP language with algebraic effects.
%
In particular, we show how to define choreographies as computations with uninterpreted effects and EPP as giving location-varying interpretations to those effects.
%
We state the correctness condition for EPP in the end.
