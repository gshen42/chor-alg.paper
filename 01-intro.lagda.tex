\section{Introduction}

Choreographic programming~(CP)~\citep{montesi-2013, montesi-2023} is a paradigm for programming distributed applications that run on multiple nodes.
%
In CP the programmer writes one, unified program, called a \emph{choreography}, that is then compiled to individual programs for each node via a compilation step called \emph{endpoint projection} (EPP).
%
For example, the following choreography describes a distributed data processing pipeline, involving nodes Alice, Bob, and Carol, which respectively run functions $f$, $g$, and $h$ on their input:
%
\begin{equation*}
  \begin{split}
    x \leftarrow \; & \locally{\mathrm{Alice}}{\mathsf{getInput}} \\
    y \leftarrow \; & \comm{\mathrm{Alice}}{\mathrm{Bob}}{f(x)} \\
    z \leftarrow \; & \comm{\mathrm{Bob}}{\mathrm{Carol}}{g(y)} \\
    w \leftarrow \; & \comm{\mathrm{Carol}}{\mathrm{Alice}}{h(z)} \\
                    & \locally{\mathrm{Alice}}{\mathsf{showResults}(w)} \\
  \end{split}
\end{equation*}
%
Here, we use $\leftarrow$ for variable bindings;
%
$\locally{\mathrm{Alice}}{t}$ denotes a local computation $t$ at Alice;
%
and $\comm{\mathrm{Alice}}{\mathrm{Bob}}{t}$ denotes communication from Alice to Bob with message $t$.
%
In this choreography, Alice first gets some input locally, processes it with $f$, and passes the result to Bob, who processes it with $g$ and passes the result to Carol, who processes it with $h$ and passes it back to Alice to be displayed to the user.
%
To get an executable program for each node, we can apply endpoint projection to the choreography, resulting in individual programs for Alice, Bob, and Carol:
%
\hspace{-1cm}
\[
\begin{minipage}{.35\textwidth}
  \begin{equation*}
    \begin{split}
      x \leftarrow \; & \mathsf{getInput} \\
                      & \mathsf{send}(\mathrm{Bob}, f(x)) \\
      w \leftarrow \; & \mathsf{recv}(\mathrm{Carol}) \\
                      & \mathsf{showResults}(w)
    \end{split}
  \end{equation*}
\end{minipage}
\hfill\vline\hfill
\begin{minipage}{.3\textwidth}
  \begin{equation*}
    \begin{split}
      y \leftarrow \; & \mathsf{recv}(\mathrm{Alice}) \\
                      & \mathsf{send}(\mathrm{Carol}, g(y)) \\
    \end{split}
  \end{equation*}
\end{minipage}
\hfill\vline\hfill
\begin{minipage}{.3\textwidth}
  \begin{equation*}
    \begin{split}
      z \leftarrow \; & \mathsf{recv}(\mathrm{Bob}) \\
                      & \mathsf{send}(\mathrm{Alice}, h(z)) \\
    \end{split}
  \end{equation*}
\end{minipage}
\]
%
A correct CP language guarantees soundness and completeness of EPP, which further implies that the collection of projected programs is deadlock-free when running together.
%
Existing research places CP on a solid theoretical foundation~\citep{montesi-2013, cruzfilipe-2020, cruzfilipe-2022, hirsch-2022},
%
which has informed the design of practical, full-featured standalone CP languages such as Choral~\citep{giallorenzo-2024}.

Recent work has introduced \emph{library-level} CP languages~\citep{shen-2023, kashiwa-2023}, in which choreographies and EPP are completely expressed as constructs in an existing host language.  For example, HasChor~\citep{shen-2023}, implements support for CP by means of a domain-specific language embedded in Haskell.
In HasChor, choreographies are monadic computations in which choreographic operators such as $\locally{\_}{\_}$ and $\comm{\_}{\_}{\_}$ may be used, and EPP is carried out by means of \emph{dynamic interpretation} of choreographies at run time.  The recently proposed ChoRus library for choreographic programming in Rust~\citep{kashiwa-2023} takes a similarly dynamic approach.
%
Library-level CP frameworks have the potential to improve the accessibility and practicality of CP by integrating it into general-purpose programming languages.
%
However, there are no proofs of correctness of EPP for library-level CP frameworks. Indeed, it is unclear to what extent the established theory of CP is applicable in the setting of library-level CP.

To close this gap, in this paper we propose \emph{algebraic effects}~\citep{plotkin-2003, plotkin-2013} as a foundational approach for implementing and verifying library-level CP.
%
Algebraic effects provide an abstraction that generalizes existing approaches to library-level CP.
%
In particular, they allow us to define choreographies as computations with user-defined effects and EPP as location-specific effect handlers.
%
Algebraic effects also lend themselves to proofs of correctness.
%
They provide abstract syntax trees for choreographies in which CP-specific effects and control flows are manifest, enabling reasoning about them.
%
Furthermore, given that algebraic effects are ``going mainstream''~\citep{dagstuhl-alg-effects-report}, with efficient implementations now available in languages such as OCaml~\citep{sivaramakrishnan-ocaml-effect-handlers} and Koka~\citep{leijen-2017}, we believe our proposed approach would make library-level CP less ad-hoc and bring it to a broader audience.

In the rest of the paper, we set up a framework for programming with algebraic effects in Agda (\Cref{sec:alg-eff-agda}), which we then use to implement a prototype library-level CP framework (\Cref{sec:cp-alg-eff}).  Finally, we discuss our ongoing work on leveraging our approach to prove the correctness of our library-level CP implementation (\Cref{sec:next}).  This paper is a literate Agda program.



