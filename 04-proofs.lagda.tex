\section{Proofs of Correctness (Work in Progress)}

\lk{We will have maybe room for 2 paragraphs at the end of p. 6 for everything we want to say about next steps.  No room for code.  It'll need to just be a few sentences.}

\lk{An obvious next step is to leverage the algebraic-effects-based formulation of CP presented in the last two sections and prove the correctness of EPP, as well as follow-on properties such as deadlock freedom.}

\lk{Given a choreographic semantics ⇒ᶜ and a network semantics ⇒ⁿ, \emph{completeness} of EPP would say that \lk{insert prose explanation}, that is, \lk{insert math}, while \emph{soundness} of EPP would say that \lk{insert prose explanation}, that is, \lk{insert math}.  However, this correctness condition is very strong, and may be too strong for our setting.  We are working on defining an appropriate semantics ⇒ᶜ and ⇒ⁿ and notions of soundness and completeness that are appropriate for our setting.}

\lk{TODO: what about deadlock freedom?}

In this section, we define the correctness condition for our CP language.
%
The goal is to show a form of semantic equivalence that EPP preserves and reflects the semantics of the original choreography.
%
Traditionally, this form of semantic equivalence is decomposed into two separate theorems --- soundness and completeness.
%
The exact definitions and proofs are still work in progress, here we only show their types to demonsrtate the key idea.
%
Note that we are not proposing anything new here.
%
Instead, we are merely showing that our algebraic effects approach is capable of doing the standard proofs.

\begin{code}[hide]
open import 02-algeff

module 04-proofs (𝕃 : Sig) where

open import Effect.Monad using (RawMonad)
open import Level using (Level)

import 03-choreo
open module 03-choreo′ = 03-choreo 𝕃
open 03-choreo′.Process
\end{code}

\subsection{Semantics of Choreographies and Networks}

We first define operational semantics for choreographies and networks.
%
The operational sematnics for choreographies treats all located values as normal values and has the following definition:
%
\begin{center}\begin{code}
opaque
  noAt : At
  noAt A _ = A

data _⇒ᶜ_ {A : Set} : Term (ℂ noAt) A → Term (ℂ noAt) A → Set where
-- work in progress
\end{code}\end{center}
\begin{code}[hide]
instance
  opaque
    unfolding noAt
    noAt-monad : ∀ {ℓ} {l} → RawMonad {ℓ} (λ A → noAt A l)
    noAt-monad = mkRawMonad _ (λ x → x) (λ x f → f x)
\end{code}
%
A network is a mapping from locations to processes, and a partially applied EPP interprets a choreography as a network.
%
Networks have the following operational semantics:
%
\begin{center}\begin{code}
data _⇒ⁿ_ {G : Loc → Set} :
  ((l : Loc) → Term ℙ (G l)) → ((l : Loc) → Term ℙ (G l)) → Set where
-- work in progress
\end{code}\end{center}
%
Note that each process returns a value of type \textit{G \ l}, which allows each location to have a different return value.

\subsection{Soundness and Completeness}

Here is the standard definition of soundness and completeness:
%
\begin{center}\begin{code}
postulate
  soundness : ∀ {F} (c c′ : Choreo F) →
    epp c ⇒ⁿ epp c′ → c ⇒ᶜ c′

  completeness : ∀ {F} (c c′ : Choreo F) →
    c ⇒ᶜ c′ → epp c ⇒ⁿ epp c′
\end{code}\end{center}
%
Soundness (Completeness) says that the network preserves (reflects) the semantics of the original choreography.
%
We note that this is a form of bisimulation where the two systems are related in locked steps.
